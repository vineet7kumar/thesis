\matlab  builtin methods are the core of the language and one of the
features that make it popular among scientists. They provide a huge set
of commonly used numerical functions.  All the operators, including the
standard binary operators (\verb|+, -, *,/|), comparison operators
(\verb|<, >, <=, >=, ==|) and logical operators (\verb+&, &&, |, ||+)
are merely syntactic sugar for corresponding builtin methods that take
the operands as arguments. For example an expression like \verb|a+b| is
actually implemented as \verb|plus(a,b)|. An important thing to note is
that unlike most programming languages, all the \matlab builtin methods
by default operate on matrix values as a whole. For example \verb|a*b|
or \verb|mtimes(a,b)| actually performs matrix multiplication on matrix
values \verb|a| and \verb|b|.  However, most of the builtin methods also
accept one or more scalar, or more accurately, $1\times1$ matrix
arguments.
Builtin methods are overloaded to accept almost all possible shapes of
arguments. Thus \verb|mtimes(a,b)| can have both \verb|a| and \verb|b| as
matrix arguments (including $1\times1$ matrices) with number of columns in
\verb|a| equal to number of rows in \verb|b|, in which case the result
is a matrix multiplication of \verb|a| and \verb|b| or one of them can
be a $1\times1$ matrix and other can be a matrix of any size and the
result is a matrix containing each element of the non-scalar argument
times the scalar argument.  Wherever possible, \matlab builtins also
support complex numerical values.  \xten on the other hand, like most of
the programming languages operates on scalar values by default.
 
Due to the fact that \xten is still new and evolving, it has a very
limited set of libraries, specially to support a large subset of
available \matlab builtin methods. The \xten Global Matrix Library (GML)
supports double-precision matrix operations however it is still not as
extensive as \matlab's set of operations and it poses some restrictions:

\begin{enumerate}

\item It works on values of type \verb|Matrix| instead of \xten type
\verb|Array| which means it needs explicit conversion of \verb|Array|
values to \verb|Matrix| values before performing a matrix operation and
and then a conversion of the results back to \verb|Array| type. This
conversion may be  a large overhead, especially for small data sizes.

\item GML is limited to Matrix values of two dimensions and containing
elements of type \verb|Double|, whereas many \matlab builtin methods
support values of greater number of dimensions.

\item GML currently does not support complex numerical values whereas
\matlab naturally supports them.

\item Currently GML requires a separate installation and configuration which
is non-trivial specially for scientists who need something that works
out of the box.

%\item TODO:verify if GML provides overloaded methods for scalars

\end{enumerate}

Due to above restrictions, \xten Global Matrix Library is useful in some
situations, for example when there is a matrix multiplication of a very
large data size, but cannot be used or is not a good choice for a large
number of operations.

For a language with open-sourced libraries, it would be possible to actually
compile the library methods to \xten.  However,  many \matlab libraries 
are closed source and thus it is not possible to translate them to \xten.

\subsection{\mixten builtin support framework}

We decided to write our own \xten implementations of the commonly used
\matlab builtin methods. Currently we have implemented
only those methods that are used in our benchmarks. In this thesis, we
concentrate on how these methods are included in the generated \xten
code with minimal loss of readability and performance rather than the
actual implementation.

The code below shows the \xten code for the \matlab builtin method
\verb|plus(a,b)|.

\noindent
\begin{lstlisting}[language=X10,numbers=none]
public static def 
  plus(a: Array[Double], b:Array[Double])
	{a.rank == b.rank}{
		val x = new Array[Double](a.region);
		for (p in a.region){
			x(p) = a(p)+ b(p);
		}
  	return x;
}
	
public static def plus(a:Double, b:Array[Double]){
	val x = new Array[Double](b.region);
	for (p in b.region){
		x(p) = a+ b(p);
	}
	return x;
}
	
public static def plus(a:Array[Double], b:Double){
	val x = new Array[Double](a.region);
	for (p in a.region){
		x(p) = a(p)+ b;
	}
	return x;
}
	
public static def plus(a:Double, b:Double){
	val x: Double;
	x = a+b;
	return x;
}
\end{lstlisting}

This \xten code contains four overloaded versions (and it still does not
contain methods to support complex values and of types other than Double) based 
on whether the arguments are scalar or array and their relative position
in the list of arguments. 

Including all the overloaded versions in the generated \xten code would result
in lot of lookup overhead, would require producing redundant code (versions
of methods with arguments of similar shape but different types will have the
same algorithm) and would generate large code with less readability. Instead we
designed a specialization technique that selects the appropriate versions of
only the methods used in the source \matlab program. Note that the use of
generic types to handle arguments of different types is not always a good idea,
since several builtin implementations involve calls to \xten library functions
which are not defined on generic types. For example, functions in the
\texttt{x10.lang.Math} librarry like \texttt{floor(Double a)},
\texttt{max(Double a, Double b)}, etc. do not take generic type arguments. 

After studying numerous builtin methods we categorized them into the
following five categories:

\begin{description}
\item[Type 1:] All the parameters are scalar values or no parameters.
\item[Type 2:] All the parameters are arrays.
\item[Type 3:] First parameter is scalar, rest of the parameters are arrays.
\item[Type 4:] Last parameter is scalar, rest of the parameters are arrays.
\item[Type 5:] Any other type(Default).
\end{description}

Each of these categories, except \emph{Type 5}, uses similar code template for
different types of values. Note that due to the three address code-like
structure of Tame IR, any call to a builtin almost always contains zero, one or
two arguments. For builtin calls like \texttt{horzcat} and \texttt{vertcat}
which may contain variable number of arguments, \mixten packs all the arguments
in a \texttt{Rail} and passes a single argument of type \texttt{Rail}.
Accordingly, the builtin is implemented in \emph{Type 2} category and receive a
single argument of type \texttt{Rail}.

We build an XML file that contains the method bodies for each category for
every builtin method (that we support). The XML also contains specialized
implementations of every builtin for different kinds of arrays.  We implement
the following strategy to select and generate the correct and required methods.
First, we make a pass through the AST to make a list of all the builtin methods
used in the source \matlab program. Next, we parse the XML file once and read
in the \xten code templates for all the categories of the builtin methods
collected in the first step. Next, whenever a call to a builtin method is made,
based on the results of the value analysis we: (1) Identify the required
specialization for the method (simple array or region array); and (2) generate
the correct method header and select the corresponding builtin template in the
required specialization for that method.  The generated methods are finally
written to a \xten class file named \verb|Mix10.x10|. In the code generated for
actual \matlab program the call to a builtin method is simply replaced by a
call to the corresponding method in the Mix10 class. For example, \matlab
expression \verb|plus(a,b)| is translated to \xten expression
\verb|Mix10.plus(a,b)|. Appendix \ref{chap:Builtinxml} demonstrates the
structure of the builtin XML with an example implementation of the builtin
\texttt{plus}. 

Using the above approach not only improves the readability of the generated
code, but it also allows for future extensibility, better maintenance and more
specialization. One of the specialization that we plan to add in future is the
ability to use the Global Matrix Library for the available methods in it and
whenever the data size is large enough. We also encourage advanced users to
mdoify the generated \texttt{Mix10.x10} file to enhance or add builtin
implementations for higher performance of the generated code. 


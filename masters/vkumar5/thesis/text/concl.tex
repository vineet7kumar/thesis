This thesis has introduced the \matlab Tamer, an extensible
object-oriented framework for supporting the translation from
dynamic \matlab programs to a Tame IR, call graph and class/type
information suitable for generating static code.  We provided an
introduction to the features of \matlab in a form that we believe
helps expose the semantics of mclasses and function lookup for
compiler and tool writers, and should help motivate some of the
restrictions we impose on the language.  We tackled the somewhat
daunting problem of handling the large number of builtin functions
in \matlab by defining an extensible hierarchy of builtins and a small
domain-specific language to define their behavior.  We defined a Tame
IR and added functionality to \mcsaf to produce the IR and to extend
the analysis framework to handle \rednote{the} new IR nodes introduced.  We
provided an interprocedural analysis framework that allows creation of
full-program analyses of \matlab programs.  Finally, we developed an
extensible value estimation analysis that we use to provide a
callgraph constructor for \matlab programs, using the proper
lookup semantics, starting with some entry point.

\section{Future Work}

Our initial experiments with the framework are very encouraging and there
are several possible projects to continue the development of
static compilers for \matlab as part of the \mclab project.
We also hope that others will also use Tamer the framework for a
variety of static \matlab tools.


In the following we will present some ideas
for the continued development of the static portion of the \mclab framework.

%We also plan to continue developing the value analysis to add richer
%abstractions for shape and other data structure properties.  Finally, as a part
%of a larger project on benchmarking \matlab, we hope to expand our set of
%benchmarks and to further examine which features might be tamed, and to extend
%our set of automated refactorings.

The major goal of the Tamer Framework is to provide a starting point for
compiler backends targeting static programming languages. In particular,
we have developed our toolkit with compilation targeting 
 {\sc FORTRAN95} in mind. In order to actually be able to compile,
the abstract value representations need to be further refined, and the value
analysis extended. In particular, shape information for arrays is needed,
which may be dealt with in a similar way as the mclass information.
Further refinement of the value representations can improve the supported feature
set and performance. For example, having exact knowledge whether numerical
values may be real, complex or imaginary allows using complex data types only
when necessary, rather than using complex numbers by default for all values.
Advanced analyses could be used to the relationships of array-shapes and
values of variables, enabling the removal of run-time array bounds checks.
This may provide significant performance benefits.

Further work may focus around expanding the set of supported \matlab features.
Interesting may be the extension of the Tamer framework to fully support
\matlab user-defined classes, including the ``old'' semantics, the ``new''
semantics since version 7.6, and possibly \rednote{handle-classes}. The \matlab Tamer
already supports the notions of mclasses, and the overloading semantics
necessary to implement class semantics are already supported. Note that in order
to support \rednote{handle-classes}, it is not sufficient to extend the value representations -
the machinery of the analysis also has to be extended \rednote{to} capture changes of
arguments that use the reference semantics of \rednote{handle-classes}. This is also
true if the Tame \matlab language subset was extended to support global and
persistent variables.

The Tamer framework could work together with \mclab's refactoring
tools in two ways. For one it would be possible to use the
refactorings as code transformations in a pre-processing step, to be
able to reduce/refactor some unsupported dynamic feature
of \matlab. For example, the refactoring toolkit allows
transforming \matlab scripts into \matlab functions. Another way the
refactoring tools could work together with the Tamer framework is in
an interactive fashion. A user wishing to compile a program
may find that the Tamer rejects it; the refactoring toolkit could
then step in and suggest to refactor the program in certain ways
to make it possible to compile.

Future work may advance the static compilation framework and the
notion of bridging the gap between dynamic languages and static
analyses and compilation.  The builtin framework with its approach to
allow the explicit and compact definition of flow information for functions
may lead to a general type annotation language for \matlab types, which could
be used both to type builtin functions, or to type user and library
functions with complex behavior. Static information provided by full-program
analyses using these frameworks could be used to find potential
runtime errors, and aid programmers build better and more correct programs.






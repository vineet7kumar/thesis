%\lstset{
%basicstyle=\footnotesize\ttfamily, 
%otherkeywords={>>},
%keywordstyle=\ttfamily\bfseries,
%numbers=none,
%commentstyle=\color{blue}\sffamily\itshape,
%stringstyle=\color{black}\ttfamily 
%} 

In this chapter, we describe key \xten semantics and features and contrast 
them with \matlab to help readers unfamiliar with \xten and \matlab to
have a better understanding of the \mixten compiler.   

\xten is an award winning open-source programming language being developed by
IBM Research. The goal of the \xten project is to provide a productive and
scalable programming model for the new-age high performance computing
architectures ranging from multi-core processors to clusters and
supercomputers~\cite{}. 

\xten, like Java, is a class-based, strongly-typed, garbage-collected and
object-oriented language. It uses Asynchronous Partitioned Global Space (APGAS)
model to support concurrency and distribution~\cite{}. The \xten compiler has a
native backend that compiles \xten programs to C++ and a managed backend that
compiles \xten programs to Java. 

In contrast to \xten, \matlab is a commercially-successful, proprietary
programming language that focuses on simplicity of implementing numerical
computation application~\cite{}. \matlab is a weakly-typed, dynamic language
with unconventional semantics and uses a JIT compiler backend.
It provides restricted support for high performance
computing via Mathworks' parallel computing toolbox~\cite{}. 

\section{Overview of \xten's key sequential features}

\xten's sequential core is a container-based object-oriented language that is 
very similar to that of Java or C++~\cite{}. A \xten program consists of a
collection of classes, structs or interfaces, which are the top-level compilation
units. Inheritance and subtyping are fairly conventional. \xten also provides very flexible
arrays based on ideas in ZPL~\cite{}.  
\subsection{Object-oriented features}

\subsection{Statements}

\subsection{Types}

\section{Overview of \xten's concurrency features}

\subsection{The APGAS Model}

\subsection{Async Construct}

\subsection{Finish Construct}

\subsection{Atomic and When Constructs}

\subsection{Places and At Construct}

\section{Overview of \xten's implementation and runtime}
 



\matlab is a popular numeric programming language, used by millions of
scientists, engineers as well as students worldwide\cite{MatlabGrowth}.  \matlab
programmers appreciate the high-level matrix operators,  the fact that variables
and types do not need to be declared, the large number of library and builtin
functions available, and the interactive style of program development available
through the IDE and the interpreter-style read-eval-print loop.  However, even
though \matlab programmers appreciate all of the features that enable rapid
prototyping,  their applications are often quite compute intensive and time
consuming. These applications could perform much more efficiently if they could
be easily ported to a high performance computing system.  

\xten\cite{x10}, on the other hand, is an object-oriented and statically-typed
language which uses cilk-style arrays indexed by \emph{Point} objects and
rail-backed multidimensional arrays, and has been designed with well-defined
semantics and high performance computing in mind.  The \xten compiler can
generate C++ or Java code and supports various communication interfaces
including sockets and MPI for communication between nodes on a parallel
computing system.

In this thesis we present MIX10, a source-to-source compiler that helps to
bridge the gap between MATLAB, a language familiar to scientists, and X10, a
language designed for high performance computing systems. MIX10 statically
compiles MATLAB programs to X10 and thus allows scientists and engineers to
write programs in MATLAB (or use old programs already written in MATLAB) and
still get the benefits of high performance computing without having to learn a
new language.Also, systems that use MATLAB for prototyping and C++ or Java for
production, can benefit from MIX10 by quickly convert- ing MATLAB prototypes to
C++ or Java programs via X10

On one hand, all the aforementioned characteristics of \matlab make it a very
user-friendly and thus popular application to develop software among a
non-programmer community. On the other hand, these same characteristics make
\matlab a difficult language to compile statically. Even the de facto standard,
Mathworks' implementation of \matlab is essentially an interpreter with a
\emph{JIT accelarator}\cite{matlabjit} which is generally slower than statically
compiled languages. GNU Octave, which is a popular open source alternative to
\matlab and is mostly compatible with \matlab, is also implemented as an
interpreter\cite{Octave}.  Lack of formal language specification, unconventional
semantics and closed source make it even harder to write a compiler for \matlab.
Furthermore, the use of arrays as default data type and the dynamicity of the
base types and shapes of arrays also make it harder to add support for
concurrency in a static \matlab compiler.  Mathworks' proprietary solution for
concurrency is the \emph{Parallel Computing Toolbox}\cite{pct}, which allows
users to use multicore processors, GPUs and clusters. However, this toolbox uses
heavyweight worker threads and has limited scalability.

Built on top of \mclab static analysis framework\cite{JesseThesis, TamerPaper},
\mixten, together with its set of reusable static analyses for performance
optimization and extended support for \matlab features, ultimately aims to
provide \matlab's ease of use, to benefit from the advantages of static 
compilation, and to expose scalable concurrency.  

%With \mixten, our aim is to provide \matlab's ease of use, to benefit from the
%advantages of static compilation, and to expose scalable concurrency. We have
%concentrated both on providing an efficient translation for the
%sequential core of \xten, as well as providing an effective bridge to
%the concurrency features of \xten. We have also extended the \mclab analysis
%framework\cite{JesseThesis, TamerPaper} by adding a set of reusable static
%analyses for performance optimization and extended support for \matlab features.

\section{Contributions}

The major contributions  of this thesis are as follows:

\begin{description}

\item[Identifying key challenges:] We have identified the key challenges
in performing a semantics-preserving efficient translation of \matlab to \xten.

\item[Overall design of \mixten:] Building upon the \mclab frontend and analysis
framework, we provide the design of the \mixten
source-to-source translator that includes a low-level \xten IR and a
template-based specialization framework for handling builtin operations.

\item[Static analyses:] We provide a set of reusable static analyses for
performance optimization and extended support for \matlab features. These
analyses include: (1) \emph{IntegerOkay analysis} - We provide an analysis to
automatically identify
variables that can be safely declared to be of type \texttt{Int} (or
\texttt{Long}) without
affecting the correctness of the generated \xten code. This helps to eliminate
most of, otherwise necessary, typecast operations which our experiments
showed to be a major performance bottleneck in the generated code; (2)
\emph{Variable renaming for type collision} - \matlab allows a variable to 
hold values of different
types at different points in a program. However, in statically typed
languages like \xten this behaviour cannot be supported since a 
variable's type needs to be declared statically by the programmer and cannot be 
changed at any
point in the program. We provide an analysis to identify and rename such
variables if their different types belong to mutually exclusive UD-DU webs; and (3)
\emph{isComplex value analysis} - We designed an analysis for identification of
complex numerical values in a \matlab program. This helped us to extend \mixten
compiler to also generate \xten code for \matlab programs that involve use of
complex numerical values.

\item[Code generation strategies for key language constructs:]  There
are some very significant differences between the semantics of \matlab
and \xten.  A key difference is that \matlab is dynamically-typed,
whereas \xten is statically-typed.   Furthermore, the type rules are
quite different, which means that the generated \xten code must include
the appropriate explicit type conversion rules, so as to match the
\matlab semantics.   Other \matlab features, such as multiple returns
from functions, a non-standard semantics for \texttt{for} loops, and a
very general range operator, must also be handled correctly.
\mixten not only supports all the key sequential constructs but also supports
concurrency constructs like \parfor and can handle vectorized instructions in a
concurrent fashion.
We have also designed and implemented a template-based system that allows us to
generate specialized \xten code for a collection of important \matlab builtin
operations.
  
\item[Techniques for efficient compilation of \matlab arrays:] Arrays are
the core of \matlab. All data, including scalar values are represented
as arrays in \matlab. Efficient compilation of arrays is the key for
good performance. \xten provides two
types of array representations for multidimensional arrays: (1)
Cilk-styled, region-based arrays and (2) rail-backed \emph{simple} arrays. 
We compare
and contrast these two array forms for a high performance computing
language in context of being used as a target language and provide techniques
to compile \matlab arrays to two different representations of arrays provided
by \xten.

\item[Working implementation and performance results:] We have implemented the
\mixten compiler over various \matlab compiler tools provided by the \mclab
toolkit.
In the process we also implemented some enhancements to these existing tools.
We provide performance results for different \xten backends over a set
of benchmarks and compare them with results from other \matlab compilers
including Mathworks' \matlab implementation and Octave.

\end{description}

\section{Thesis Outline}

This thesis is divided into \ref{chap:Conclusions} chapters, including this one
and is structured as follows.

\chapref{chap:Xten} provides an introduction to the \xten language and describes
how it compares to \matlab from the point of view of language design.
\chapref{chap:Design} gives a description of various existing \matlab
compiler tools upon which \mixten is implemented, presents a high-level
design of \mixten, and explains the design and need of \mixten IR.
In \chapref{chap:Analyses} we provide a description of the \emph{IntegerOkay}
analysis to identify variables that are safe to be declared as \texttt{Long}
type, \emph{variable renaming for type conflict} to rename variables with
conflicting types in isolated UD-DU webs and \emph{isComplex}
analysis to identify complex numerical values. 
\chapref{chap:Codegen} gives details of code generation strategies for
important \matlab constructs.
In \chapref{chap:Arrays} we introduce different types of arrays provided by
\xten,  we identify pros and cons of both kinds of arrays in the context of
\xten as a target language and describe code generation strategies for them.
%In \chapref{chap:Concurrency} we describe our code-generation strategies for
%existing \matlab concurrency constructs and addition of new fine-grained
%concurrency controls.
%In \chapref{chap:Analysis} we provide a detailed description of our analysis to
%identify complex numerical values in \matlab programs.
\chapref{chap:Evaluation} provides performance results for code generated using
\mixten for a suite of benchmarks.
\chapref{chap:Related} provides an overview of related work and
\chapref{chap:Conclusions} concludes and outlines possible future work.

\matlab is a popular numeric programming language, used by millions of
scientists, engineers as well as students worldwide\cite{MatlabGrowth}.  \matlab
programmers appreciate the high-level matrix operators,  the fact that
variables and types do not need to be declared, the large number of library and
builtin functions available, and the interactive style of program development
available through the IDE and the interpreter-style read-eval-print loop.
However, even though \matlab programmers appreciate all of the features that
enable rapid prototyping,  their applications are often quite compute intensive
and time consuming. These applications could perform much more efficiently if
they could be easily ported to a high performance computing system.  

On the other hand, \xten is an object-oriented and statically-typed language
which uses cilk-style arrays indexed by \emph{Point} objects, and has been
designed with well-defined semantics and high performance computing in mind.
\xten compiler can generate C++ or Java code and supports various communication
interfaces including sockets and MPI for communication between nodes on a
parallel computing system.

In this thesis we present \mixten, a source-to-source compiler that helps
to bridge the gap between \matlab, a language familiar to scientists,
and \xten,  a language designed for high performance. \mixten statically
compiles \matlab programs to \xten and thus
allows scientists and engineers to write programs in \matlab (and use old 
programs already written in \matlab) and still get the benefits of high 
performance computing without having to learn a new language. Also, systems that
use \matlab for prototyping and C++ or Java for production, can benefit from
\mixten by quickly converting \matlab prototypes to C++ or Java programs via 
\xten. In particular, this thesis identifies the key challenges in compiling a
dynamically-typed language like \matlab to a statically-typed object-orinted
high performance computing language like \xten, our approach to 
compiling \matlab to \xten.

The ultimate goal of the \mixten compiler is two-fold.  First, it can be used
as a back-end for a \matlab system,  producing high-performance code via \xten.
Second, it can be used to help programmers port their \matlab code to \xten
source code.   The techniques presented in this paper provide the core upon
which these two ultimate goals can be achieved.

\begin{comment}
\mixten is built on top of the \mclab front-end and analysis toolkits.  
- Introduce \matlab briefly, why is it difficult but important to have a
  static compiler for it and why do we choose \xten as our target.- concentrate
more on high -level issues like lack of documentation, scientists are not
programmers,performance issues, popularity, need for static compilation to
other high-level languages. We talk about \matlab semantics and wild features
in the next section\smallskip
- Briefly introduce the high-level design of \mixten.\smallskip 
\end{comment}
  
The major contributions  of this paper are as follows:

\begin{description}

\item[Identifying key challenges:] We have identified the key challenges
in performing a semantics-preserving translation of \matlab to \xten.

\item[Overall design of \mixten:] We provide the design of a 
source-to-source translator, building upon the McLab front-end and
analysis toolkits.

\item[\mixten IR design:] In order to provide a convenient
target for the first level of translation, we have defined a high-level
\mixten IR.  This IR is currently used for code generation,  but in the
future will also be used for code simplifications and transformations.

\item[Template-based builtin framework:] \matlab supports many builtin
operations that can operate over a wide variety of run-time types.  We
have designed and implemented a template-based system that allows us to
generate specialized \xten code for a collection of important builtin
operations.

\item[Code generation strategies for key language constructs:]  There
are some very significant differences between the semantics of \matlab
and \xten.  A key difference is that \matlab is dynamically-typed,
whereas \xten is statically-typed.   Furthermore, the type rules are
quite different, which means that the generated \xten code must include
the appropriate explicit type conversion rules, so as to match the
\matlab semantics.   Other \matlab features, such as multiple returns
from functions, a non-standard semantics for \texttt{for} loops, and a
very general range operator, must also be handled correctly.

\item[Working core implementation:] We have implemented the core
functionality for the \mixten compiler, concentrating on the sequential
part of \xten, and we provide some initial results.

\end{description}

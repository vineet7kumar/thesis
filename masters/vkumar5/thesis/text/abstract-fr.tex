\begin{otherlanguage}{french}
\matlab est un langage scientifique utilisé par des ingénieurs,
scientifiques, et étudiants à travers le monde. Bien que \matlab soit
très approprié pour les prototypages et les développements rapides, les
usagers veulent souvent convertir leurs programmes \matlab finaux vers
un langage statique tel {\sc Fortran}, dans le but de les intégrer à
des programmes existants dans ce langage, de tirer avantage des
performances des compilateurs statiques plus puissants, ou de faciliter
la distribution des fichiers exécutables.

Cette thèse présente un toolkit extensible orienté objet pour
faciliter la production de programmes statiques à partir de programmes
\matlab dynamiques. Notre toolkit à code source libre, appelé \matlab
  Tamer («dompteur \matlab»), vise un large sous-ensemble de
\matlab. À partir d'informations sur le point d'entrée du programme,
le \matlab Tamer construit un graphe d'appels complet, transforme
chaque fonction en une représentation réduite intermédiaire et fournit
l'information sur le typage pour faciliter la production du code
statique.

Pour fournir cette fonctionnalité, nous devons manipuler une grand
nombre de fonctions \matlab intégrées. Une partie du cadre du Tamer
est le cadre intégré, un toolkit extensible fournissant une approche
de principe pour manipuler un grand nombre de fonctions
intégrées. Pour construire le graphe d'appels, nous fournissons un
cadre d'analyse inter-procédural pouvant être utilisé pour implanter
des analyses de programmes complets. En utilisant ce cadre
inter-procédural, nous avons développé l'analyse des valeurs, une
analyse inter-procédurale extensible pour estimer les types \matlab,
pour aider à découvrir les arrêtes d'appels nécessaires pour
construire le graphe d'appels.

Pour pouvoir rendre faisable une analyse statique, nous interdisons un
petit nombre de concepts et caractéristiques de \matlab, mais nous
tentons de supporter un sous-ensemble de \matlab aussi grand que
possible. Conséquemment, en restreignant légèrement \matlab, en
fournissant un puissant cadre d'analyse et en simplifiant les
transformations, nous pouvons «dompter \matlab».

\end{otherlanguage}


The work presented in this paper provides an alternative to Mathworks' de
facto proprietary implementation of \matlab.   Our approach is open and
extensible and leverages the high-performance computing abilities of
\xten.

Although our focus is on handling \matlab itself, notable open source
alternatives of \matlab like Scilab\cite{Scilab}, Julia\cite{julia},
NumPy\cite{numpy} and Octave\cite{Octave} provide limited concurrency features.
They concentrate on providing open library support and have not tackled the
problems of static compilation.  We are investigating if there is any way of
sharing some of their library support with \mixten.  The MEGHA
project\cite{megha} provides an interesting approach to map \matlab array
operations to CPUs and GPUs, but supports only a very small subset of \matlab.  

%As discussed in \secref{Sec:Background}, this work builds upon the
%previous work from the \mclab group, including the front-end, the \mcsaf
%analysis framework~\cite{JesseThesis,McSAFecoop12} analysis framework,
%and the \matlab Tamer~\cite{TamerPaper}.

There have been  previous research projects on static compilation of
\matlab which focused particularly on the array-based subset of \matlab
and developed advanced static analyses for determining shapes and sizes
of arrays.  For example, FALCON \cite{falcon} is a \matlab to {\sc
Fortran90} translator with sophisticated type inference algorithms.
The McLab group has previously implemented a prototype Fortran 95
generator~\cite{McForThesis}, and has more recently developed the next generation
Fortran generator, \mctwofor~\cite{mc2for} in parallel with the \mixten project.   Some of the
solutions are shared between the projects, especially the parts which
extend the Tamer.  

\textit{MATLAB Coder} is a commercial compiler by MathWorks\cite{MATLABCoder},
which produces C code for a subset of \matlab. 

In terms of source-to-source compilers for \xten, we are aware of two
other projects.  StreamX10 is a stream programming framework based on
\xten~\cite{Wei-2012}.  StreamX10  includes a compiler which translates
programs in COStream to parallel \xten code.  Tetsu discusses the design
of a Ruby-based DSL for parallel programming that is compiled to
\xten~\cite{Tetsu-2011}.
